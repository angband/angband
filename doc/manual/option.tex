\section{Option Descriptions --- option.txt}
\paragraph{}Options are accessible through the '=' command, which provides an
interface to the various sets of options available to the player.

\paragraph{}In the decriptions below, each option is listed as the textual summary
which is shown on the ``options'' screen, plus the internal name of the
option in brackets, followed by a textual description of the option.

\paragraph{}Note that the internal name of the option can be used in user pref files
to force the option to a given setting, see ``customize.txt'' for more
info.

\paragraph{}Various concepts are mentioned in the descriptions below, including
``disturb'', (cancel any running, resting, or repeated commands, which are
in progress), ``flush'' (forget any keypresses waiting in the keypress
queue), ``fresh'' (dump any pending output to the screen), and
``sub-windows'' (see below).

\subsection{Interface options}
\begin{description}
\item[Use sound] \verb+[use_sound]+\\
    Turns on sound effects, if your system supports them.

\item[Rogue-like commands] \verb+[rogue_like_commands]+\\
    Selects the ``roguelike'' command set (see ``command.txt'' for info).

\item[Use old target by default] \verb+[use_old_target]+\\
    Forces all commands which normally ask for a ``direction'' to use the
    current ``target'' if there is one. Use of this option can be dangerous
    if you target locations on the ground, unless you clear them when done.

\item[Always pickup items] \verb+[pickup_always]+\\
\item[Always pickup items matching inventory] \verb+[pickup_inven]+\\
If pickup\_always is on, all items are always picked up, providing it is
safe to do so. If pickup\_inven is on, then items matching those in your
inventory are always picked up.

\item[Open/Disarm/Close without direction] \verb+[easy_open]+\\
    If there is only one closed door next to the character, the game will
    not bother asking for a direction when the player asks to ``o''pen
    something. The same applies for ``c''losing doors and ``D''isarming traps.

\item[Allow mouse clicks to move the player] \verb+[mouse_movement]+\\
    Clicking on the main window will be interpreted as a move command to
    that spot.

\item[Show mouse status line buttons] \verb+[mouse_buttons]+\\
    Creates clickable ``buttons'' on the bottom line of the main window, to
    enhance mouse-led play.

\item[Allow accents in output files] \verb+[xchars_to_file]+\\
    Extended characters (accents, umlauts and so on) will be printed in
    files created by the game (character dumps, screenshots etc.).
\end{description}

\subsection{Display options}
\begin{description}
\item[Player color indicates low hit points]
\verb+[hp_changes_color]+\\
    This option makes the player @ turn various shades of colour from white
    to red, depending on percentage of HP remaining.

\item[Center map continuously] \verb+[center_player]+\\
    The map always centres on the player with this option on. With it off, it
    is divided into 25 sections, with coordinates (0,0) to (4,4), and will
    show one section at a time - the display will ``flip'' to the next section
    when the player nears the edge.

\item[Show flavors in object descriptions] \verb+[show_flavors]+\\
    Display ``flavors'' (color or variety) in object descriptions, even for
    objects whose type is known. This does not affect objects in stores.

\item[Use special colors for torch-lit grids]
\verb+[view_yellow_light]+\\
    This option causes special colors to be used for ``torch-lit'' grids in
    certain situations (see ``view\_granite\_light'' and
    ``view\_special\_light'').
    Turning this option off will slightly improve game speed.

\item[Animate multi-coloured monsters and items]
\verb+[animate_flicker]+\\
    Certain powerful monsters and items will shimmer in real time, i.e.
    between keypresses.

\item[Show unique monsters in a special colour]
\verb+[purple_uniques]+\\
    All ``unique'' monsters will be shown in a light purple colour, which
    is not used for any ``normal'' monsters --- so you can tell at a glance
    that they are unique. If you like the idea but don't like the colour,
    you can edit it via the ``interact with colors'' option.
\end{description}

\subsection{Disturbance options}
\begin{description}
\item[Disturb whenever any monster moves] \verb+[disturb_move]+\\
    Disturb the player when any monster moves, appears, or disappears.
    This includes monsters which are only visible due to telepathy, so
    you should probably turn this option off if you want to ``rest'' near
    such monsters.

\item[Disturb whenever viewable monster moves] \verb+[disturb_near]+\\
    Disturb the player when any viewable monster moves, whenever any
    monster becomes viewable for the first time, and also whenever any
    viewable monster becomes no longer viewable. This option ignores
    the existance of ``telepathy'' for the purpose of determining whether
    a monster is ``viewable''. See also the ``view\_reduce\_view'' option.

\item[Disturb whenever leaving trap detected area]
\verb+[disturb_detect]+\\
    This option causes you to be disturbed when you reach the edge of your
    last invocation of detect traps.

\item[Disturb whenever player state changes] \verb+[disturb_state]+\\
    This option causes you to be disturbed whenever the player state
    changes, including changes in hunger, resistance, confusion, etc.

\item[Automatically clear --more-- prompts] \verb+[auto_more]+\\
    The game does not wait for a keypress when it comes to a --more--
    prompt, but carries on going.

\item[Notify on object recharge] \verb+[notify_recharge]+\\
    This causes the game to print a message when any rechargeable object
    (i.e. a rod or activatable weapon or armour item) finishes recharging.
    It is the equivalent of inscribing {!!} on all such items.
\end{description}

\subsection{Birth options}
\begin{description}
\item[Maximize effect of race/class bonuses] \verb+[birth_maximize]+\\
     Normally, if a character should find potions that permanently increase
     his stats, the maximum that can be achieved without equipment bonuses
     is 18/100. If maximize mode is on (which is the default), then the race
     and class bonuses and penalties apply to this limit, so your half-troll
     will have a greater maximum possible strength (unassisted by his
     equipment) than a hobbit. With maximize mode off, all characters max
     out at 18/100 for all stats. This also affects the start of the game,
     because race and class bonuses are applied as if they were
     ``rings'' in
     maximize mode (at an exact rate of 0/10 percentile points equals 1 point
     for stats over 18), and as ``potions'' in non-maximize mode (drinking six
     potions of [stat] will probably increase your stat by more than 0/60 if
     it was at 18 to start with.) To cut a long story short and a confusing
     explanation simple, you get better starting stats in non-maximize mode,
     but can't get as good at what you're supposed to be best at in the later
     stages of the game.

\item[Randomize the artifacts (except a very few)]
    \verb+[birth_randarts]+\\
     A different set of artifacts will be created, in place of the standard
     ones. This is intended primarily for people who have played enough to
     know what most of the standard artifacts do and want some variety. The
     number, findability and power of random artifacts will all match the
     standard set --- approximately.

\item[Use previous set of randarts] \verb+[birth_keep_randarts]+\\
     Does what it says on the tin. Note that this option must be set at the
     END of the game, from the death menu. Once you have restarted the game,
     it is too late to set it (but you can still un-set it if you change your
     mind). Note that switching to non-random artifacts will wipe randarts,
     regardless of this option.

\item[Monsters chase recent locations] \verb+[birth_ai_smell]+\\
     Allow monsters to take advantage of ``old'' trails that you may have left
     in the dungeon. This has no effect unless ``flow\_by\_sound'' is also set.

\item[Monsters act smarter in groups] \verb+[birth_ai_packs]+\\
     ``Group'' monsters will use tactics that groups of monsters might
     reasonably employ --- such as hiding out of sight around a corner,
     trying to draw the character out into the middle of a room so they
     can surround him and all attack at once, rather than chasing one
     by one down a corridor.

\item[Monsters learn from their mistakes] \verb+[birth_ai_learn]+\\
     Allow monsters to learn what spell attacks you are resistant to,
     and to use this information to choose the best attacks.

\item[Monsters exploit players weaknesses] \verb+[birth_ai_cheat]+\\
     Allow monsters to know what spell attacks you are resistant to,
     and to use this information to choose the best attacks.

 \item[Monsters behave more intelligently (broken)]
 \verb+[birth_ai_smart]+\\
     Monsters have a certain amount of intelligence and will use it ---
     teleporting away when low on hit points, using their more powerful
     spells in preference to the less powerful spells, etc. Not all
     monsters are, of course, intelligent --- but those who are are almost
     unkillable with this option on (especially if they can heal
     themselves).

\item[Restrict the use of stairs/recall] \verb+[birth_ironman]+\\
     This option, not recommended for non-advanced players, prevents the
     generation of up staircases, and makes the scroll and effect of Word
     of Recall not function: and teleport-level always goes down, as does
     the spell of stair-creation, except on the quest levels of Sauron and
     Morgoth.

\item[Restrict the use of stores/home] \verb+[birth_no_stores]+\\
     The stores are all closed. The home is someone else's, and locked. You
     can keep nothing but what you carry with you, and get nothing but what
     you find in the dungeon. No selling items, or buying potions of
     restore stat\ldots Not recommended for new players, or indeed for
     sane players.

 \item[Restrict the creation of artifacts]
 \verb+[birth_no_artifacts]+\\
     No artifacts will be created. Ever. Just *how* masochistic are you?

\item[Don't stack objects on the floor] \verb+[birth_no_stacking]+\\
     Items dropped on the floor will spread out instead of stacking.
     Normal items will disappear if there is no empty grid within a radius
     of three squares.

\item[Lose artifacts when leaving level] \verb+[birth_no_preserve]+\\
     Normally if you leave a level with an unidentified artifact on it
     you may still find it later. With this option on, if you leave a level
     with an artifact on it's gone for the rest of the game whether you
     knew it was there or not. Note that this option has no effect on
     artifacts which you have already identified (i.e. picked up) --- these
     will always be permanently lost if you leave a level without taking
     them with you.

\item[Don't generate connected stairs] \verb+[birth_no_stairs]+\\
     Never generate a staircase back to the level you came from, if you used
     a staircase to get to the level.

\item[Don't show level feelings] \verb+[birth_no_feelings]+\\
     The ``quality'' of a level is never revealed, no matter how long you
     spend on the level.

\item[Items always sell for 0 gold] \verb+[birth_no_selling]+\\
     Shopkeepers will never pay you for items you sell, though they will
     still identify unknown items for you, and will still sell you their
     wares. To balance out income in the game, gold found in the dungeon
     will be increased if this option is on.
\end{description}

\subsection{Cheating options}
\begin{description}
\item[Peek into monster creation] \verb+[cheat_hear]+\\
    Cheaters never win. But they can peek at monster creation.

\item[Peek into dungeon creation] \verb+[cheat_room]+\\
    Cheaters never win. But they can peek at room creation.

\item[Peek into something else] \verb+[cheat_xtra]+\\
    Cheaters never win. But they can see debugging messages.

\item[Know complete monster info] \verb+[cheat_know]+\\
    Cheaters never win. But they can know all about monsters.

\item[Allow player to avoid death] \verb+[cheat_live]+\\
    Cheaters never win. But they can cheat death.
\end{description}

\subsection{Window flags}
\paragraph{}Some platforms support ``sub-windows'', which are windows
which can be used to display useful information generally available
through other means. The best thing about these windows is that they are
updated automatically (usually) to reflect the current state of the
world. The ``window options'' can be used to specify what should be
displayed in each window. The possible choices should be pretty
obvious.

\begin{description}
\item[Display inven/equip]
    Display the player inventory (and sometimes the equipment).

\item[Display equip/inven]
    Display the player equipment (and sometimes the inventory).

\item[Display player (basic)]
    Display a brief description of the character, including a breakdown
    of the current player ``skills'' (including attacks/shots per round).

\item[Display player (extra)]
    Display a special description of the character, including some of the
    ``flags'' which pertain to a character, broken down by equipment item.

\item[Display player (compact)]
    Display a brief description of the character, including a breakdown
    of the contributions of each equipment item to various resistances
    and stats.

\item[Display map view]
    Display the area around the player or around the target while
    targetting. This allows using graphical tiles in their original
    size.

\item[Display messages]
    Display the most recently generated ``messages''.

\item[Display overhead view]
    Display an overhead view of the entire dungeon level.

\item[Display monster recall]
    Display a description of the most monster which has been most recently
    attacked, targetted, or examined in some way.

\item[Display object recall]
    Display a description of the most recently selected object. Currently
    this only affects spellbooks and prayerbooks. This window flag may be
    usefully combined with others, such as ``monster recall''.

\item[Display monster list]
    Display a list of monsters you know about and their distance from you.

\item[Display status]
    ???

\item[Display item list]
    Display a list of items you know about and their distance from you.
 
\item[Display borg messages]
    This window flag is currently used only by the Borg.

\item[Display borg status]
    This window flag is currently used only by the Borg.
\end{description}

\subsection{Left Over Information}
\paragraph{}The ``hitpoint\_warn'' value, if non-zero, is the percentage
of maximal hitpoints at which the player is warned that he may die. It
is also used as the cut-off for using the color red to display both
hitpoints and mana.

\paragraph{}The ``delay\_factor'' value, if non-zero, will slow down the
visual effects used for missile, bolt, beam, and ball attacks. The
actual time delay is equal to ``delay\_factor'' squared, in
milliseconds.

